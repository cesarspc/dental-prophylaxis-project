\documentclass[conference]{IEEEtran}
\IEEEoverridecommandlockouts
% The preceding line is only needed to identify funding in the first footnote. If that is unneeded, please comment it out.
\usepackage{cite}
\usepackage{amsmath,amssymb,amsfonts}
\usepackage{algorithmic}
\usepackage{graphicx}
\usepackage{textcomp}
\usepackage{xcolor}
\def\BibTeX{{\rm B\kern-.05em{\sc i\kern-.025em b}\kern-.08em
    T\kern-.1667em\lower.7ex\hbox{E}\kern-.125emX}}
\begin{document}

\title{Dental prophylaxis domain-specific LLM\\

}

\author{\IEEEauthorblockN{1\textsuperscript{st} Catalina Ariza}
\IEEEauthorblockA{\textit{Ingeniería de Sistemas} \\
\textit{Universidad Distrital Francisco José de Caldas}\\
Bogotá, Colombia \\
catarizaa@udistrital.edu.co}
\and
\IEEEauthorblockN{2\textsuperscript{nd} Cesar Pulido}
\IEEEauthorblockA{\textit{Ingeniería de Sistemas} \\
\textit{Universidad Distrital Francisco José de Caldas}\\
Bogotá, Colombia \\
capulidoc@udistrital.edu.co}
}

\maketitle

\begin{abstract}
    Oral care is a subject that many people are interested in, either to take care of or improve their hygiene, to know and apply better habits for their health or to know if they have any disease or cavities that should be treated. The branch of dentistry that studies and treats all these things is dental prophylaxis, which is a set of preventive measures aimed at preventing oral diseases. In order to be able to help people solve their questions quickly regarding this topic, we will make use of a chat bot, which will have the most relevant information about dental prophylaxis care and advice. 
\end{abstract}

\begin{IEEEkeywords}
prophylaxis, AI, dentistry
\end{IEEEkeywords}

\section{Introduction}
Dental prophylaxis is a topic of interest for many people, since it has to do directly with health care, more specifically that of the mouth; no one wants to have poor oral health because as a result of the problems that can be generated in the mouth, these can escalate to problems in the rest of the body, even the simple fact of being able to eat properly can be affected. The biggest problem with dental health in people is that they do not have a good source of information or they are simply lazy to do research on the subject. Because of this, making a chatbot that provides fast, accurate and precise information on this topic can be very useful.

For the realization of the chatbot, the functioning of the LLaMa model and its easy implementation with fine-tuning tools will be used as a reference. In addition, for the systemic analysis we will take into account the general theory of systems and different examples of analysis already performed to other systems by other people, such as a soccer team or the solar system. 

After reviewing these aspects, the first thing we did was an investigation on dental prophylaxis, where we found that its objective is the prevention of oral diseases through the use of dental materials that are simple to use for common people and also procedures performed by dentists \cite{b1}\cite{b2}. 

At the time of analyzing this system, two important subsystems were defined, the personal care of the patient and the professional procedures, as well as the elements of these and their relationships, also the inputs and outputs of the system, together with the feedback it receives (See Figure 1).

Once this diagram and its components were made, we began to study it, applying different concepts, such as synergy, holism, homeostasis, chaotic attractors, among others, as well as the different types of analysis that can be implemented, such as sensitivity and complexity analysis. This with the objective of understanding the system better and giving more correct and informed information to the end user.

\begin{figure}[htbp]
    \centerline{\includegraphics[width=0.46\textwidth]{figures/fig1.png}}
    \caption{System diagram.}
    \label{fig}
\end{figure}

\section{Method and materials}

With this project we can help people to solve their doubts regarding their oral health and begin to apply good dental cleaning habits, also this can be a benefit not only for the health of the person using the chatbot, but also for the dental health system, because it would facilitate the processes to be performed to each patient for their good cleaning practices.

For the implementation and training of the model, we will use the open-source LLaMa project, which is highly customizable and tunable to meet the requirements of the project. We will be able to adjust specific performance parameters for the needs of the specific domain we want to give it. For this purpose we will have a dataset of about 200 files in pdf format with information oriented to the topic.

\subsection{Technical decisions}

Dental prophylaxis also studies the external causes of why periodontal diseases or caries can occur; this includes the investigation of bad eating habits and vices, however, in order to narrow down the problem so that the analysis does not extend too far, we decided to leave the external causes aside and focus only on the personal care that each person should follow plus the procedures that the dentist performs to prevent or improve these oral diseases.

\begin{thebibliography}{00}
\bibitem{b1} A. Barrell, “What is dental prophylaxis?,” Medicalnewstoday.com, Dec. 08, 2022.
\bibitem{b2} C. Gross, “Discover the importance of dental prophylaxis with Gross Dentists,” Gross Dentistas, May 18, 2023.
\end{thebibliography}

\end{document}