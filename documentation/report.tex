\documentclass[12pt]{article}
\usepackage{graphicx}
\usepackage{geometry}
\usepackage{fancyhdr}
\usepackage{indentfirst}
\geometry{margin=1in}
\setlength{\parskip}{0.6em}
\setlength{\parindent}{1.5em}

\pagestyle{fancy}
\fancyhf{}
\fancyhead[L]{Dental Prophylaxis Chatbot}
\fancyfoot[C]{\thepage}

\title{Dental prophylaxis chatbot - Project CatchUp \\ \large Systems Analysis \\ \large Semester 2024-III}
\author{Cesar Augusto Pulido Cuervo \and Catalina Ariza Ardila}
\date{}

\begin{document}

\maketitle

\section*{System}

The dental prophylaxis is a very extensive subject, although it may not seem so at first. Prophylaxis refers to measures taken to prevent medical conditions that may occur. Dental prophylaxis is a odontology branch that include medications, cleanings and treatments that are aimed at preventing dental infections, illnesses, or other health problems before they arise.

Multiple elements of the dental prophylaxis system cooperate to guarantee the best possible oral health. The first step in the system is patient care, which includes brushing, flossing, and mouthwash as necessary practices to keep teeth clean and prevent dental diseases. Professional practices including fluoride treatment, dental sealants, and regular cleanings by a dentist support these measures. Dental sealants function as protective barriers, keeping food and germs from building up in sensitive places, while fluoride helps to strengthen teeth and lower the chance of cavities.

When combined, these preventive steps ensure improved oral health status by addressing common oral health issues such as periodontal diseases and cavities as shown in Figure \ref{fig:system}. The feedback loop illustrates how dentist interventions combined with routine updates to a patient's medical history and supplies lead to ongoing improvements in oral care techniques.

Interesting questions to be solved by the bot:

\begin{itemize}
    \item What do you recommend me to have a better oral care?
    \item Give me 10 recommendations to maintain oral health.
    \item When should I go to the dentist?
    \item How long should I wash my mouth with my toothbrush?
    \item What happens if I don't use mouthwash?
    \item How often should I floss my teeth?
    \item What diseases can develop from not brushing my mouth?
    \item Do I have to use mouthwash every day?
    \item What do I do if I have a toothache?
    \item How do I floss my teeth?
\end{itemize}

\begin{figure}[htbp]
\centerline{\includegraphics[width=0.96\textwidth]{figures/fig1.png}}
\caption{Dental prophylaxis system.}
\label{fig:system}
\end{figure}

\begin{itemize}
    \item Patient history element can generate a butterfly effect.
    \item Any element with an error can generate a snowball or domino effect.
\end{itemize}

\section*{Bounding}
In defining the boundaries of the system, recognizing the vast scope of odontology as a field was important. Owing to its breadth, attention was given to the specialty of prophylaxis, which includes oral disease prevention strategies. Daily routines like eating habits have a big impact on dental health, but adding these aspects would make the system more complex and broaden its use needlessly. Because of this, the system is designed with the specific goal of focusing only on the professional interventions and basic self-care with dental health materials associated with dental prophylaxis, making it possible to avoid oral illnesses in a more efficient and effective manner.

The following aspects are the inputs of the system:

\begin{itemize}
    \item \textbf{Oral care materials: }What the patient have for them dental self-care.
    \item \textbf{Patient history: }Previous treatments, allergies or any ongoing medical conditions.
    \item \textbf{Dentist: }What professional or institution is attending the patient, their tools, the quality of the patient's care.
    \item \textbf{Medications: }Pharmaceutical products that preserve oral health.
\end{itemize}

As the output is the health status, this feedback will lead to what medications the patient should take or if the patient's history has been updated.

\section*{Complexity Analysis}

The dental prophylaxis system is highly interdependent among its constituent parts, which contributes to its high level of complexity. In the event that a patient misses regular dental checkups, the dentist may find it more difficult to provide the necessary care. Because of their interconnectedness, chaotic attractors can have an unforeseen impact on a wide range of system components.


\section*{Sensitivity analysis}
\subsection*{Snowball effect}
The snowball effect apears when in a small issue with an element escalates into a major problem. In the prophylaxis system, this can ocurr when:

\textbf{Improper brushing $\rightarrow$ Excess plaque $\rightarrow$ Cavities form $\rightarrow$ Oral diseases}

Is very common in many patients this progression from a minor problem, if unchecked, can lead to significant dental problems.

\subsection*{Domino effect}
The domino effect appears when a vulnerable element of the system is affected, causing a chain that involve various parts of the system. In this system, it can appear as follows:

\textbf{Untreated cavities + No flossing $ \rightarrow $ Toothache $ \rightarrow $ Periodontal diseases}

\subsection*{Butterfly Effect}
The butterfly effect refers to how small actions or inactions in the past can lead to seemingly unexpected consequences in the future. In the context of the dental prophylaxis system, this effect can occur when early preventive measures are neglected but the consequences do not manifest immediately. For example:

\textbf{Not enough dental care in childhood $\rightarrow$ No visible issues for a long time $\rightarrow$ Better dental care practices $\rightarrow$ Cavities appear spontaneously}

This effect illustrates how long-term consequences of inadequate dental care may remain hidden for years, only to surface later despite improved hygiene habits.


\end{document}
